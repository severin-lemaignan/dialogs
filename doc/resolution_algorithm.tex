\documentclass[a4paper]{article}  
\usepackage{pseudocode}

\begin{document}
\begin{pseudocode}[ruled]{Resolution}{sentence, currentSpeaker}
\label{Resolution}

\PROCEDURE{GenerateDescription}{group} 
   noun \GETS \CALL{GetNoun}{group} \\ 
   \IF \CALL{Ontology.Lookup}{noun} \in (Instances) \STMTNUM{7.5em}{st.lookup} \THEN
   		\BEGIN
		id \GETS \CALL{Ontology.lookup}{noun}\\	
		\RETURN {\mathcal{D} + \{ *\ {\tt sameAs}\ <id> \}}\\
		\END
   \ELSE
    	\mathcal{D} = \mathcal{D} + \{ *\ {\tt type}\ <noun>\} \\
   
   \\
   det \GETS \CALL{GetDeterminant}{group} \\
   \IF det \in {\mbox(possessives)} \THEN
       \mathcal{D} = \mathcal{D} + \{ *\ {\tt isRelatedTo}\ <possessor>\} \\
    
    \IF det \in {\mbox(demonstratives)} \THEN
        \BEGIN
        \IF \CALL{Ontology.Check}{\{<currentSpeaker>\ {\tt focusesOn}\ *\}} \THEN 
            \mathcal{D} = \mathcal{D} + \{<currentSpeaker>\ {\tt focusesOn}\ *\}
        \ELSE
            \mathcal{D} = \mathcal{D} + \CALL{AnaphoricMatching}{} \STMTNUM{4em}{st.anaphoric} \\
        \END \\
   \\
   adjs \GETS \CALL{GetAdjectives}{group} \\
   \FOREACH adj \in adjs \DO
   	\BEGIN
   		\IF adj == <other> \THEN 
   			\BEGIN
   			id \GETS \CALL{History.GetMatchingGroup}{group} \STMTNUM{8em}{st.history}\\
   			\mathcal{D} = \mathcal{D} + \{ *\ {\tt differentFrom}\ <id> \}\\
   			\RETURN{D}\\
			\END   		
   		\ELSE
	     	\mathcal{D} = \mathcal{D} + \{ *\ {\tt hasFeature}\ <adj>\} \STMTNUM{9em}{st.adj} \\
    \END\\
    
   \\  
   nounComplements \GETS \CALL{GetNounComplements}{group} \\
   \FOREACH nouncmpl \in nounComplements \DO
     \mathcal{D} = \mathcal{D} + {\CALL{GenerateDescription}{nouncmpl}}\\
   
   
   \\  
   relativeClauses \GETS \CALL{GetSubordinateRelativeClauses}{group} \\
   \FOREACH relative \in relativeClauses \DO
   	\BEGIN
   	 \mathcal{G} \GETS \CALL{GetNominalGroups}{relative} \\
   	 \FOREACH g \in \mathcal{G} \DO
     	\mathcal{D} = \mathcal{D} + {\CALL{GenerateDescription}{g}}
    \END\\
     
   \\
   \RETURN{\mathcal{D}} 
\ENDPROCEDURE

\label{history}
\PROCEDURE{History.GetMatchingGroup}{group}
\COMMENT{Extract Nominal group from sentences stored in the history}\\
\mathcal{H} \GETS \CALL{History.GetAllNominalGoup}{}\\
\COMMENT{Generate description of the nominal group that is being processed.
 The adjective  "other" is to be removed before calling this routine}	
	\mathcal{G} \GETS \CALL{GenerateDescription}{group} \\ 
	
	candidates \GETS \mathcal{H} \cap \mathcal{G}\\
	\IF \left|{candidates}\right| = 0 \THEN
    \BEGIN
      \OUTPUT{\mbox{Couldn't find another object with the same characteristics!}} \\
      \EXIT \\
    \END
   \ELSEIF \left|{candidates}\right| = 1 \THEN
      id \GETS candidates[0]
   \ELSE
   	  id \GETS \CALL{Discrimination}{candidates}\\
   \RETURN{id}
\ENDPROCEDURE


\MAIN

\COMMENT{Extract nominal groups from the sentence} \\
\mathcal{G} \GETS \CALL{ParseNominalGroups}{sentence} \\

\FOREACH g \in \mathcal{G} \DO 
\BEGIN
   \mathcal{D} \GETS \CALL{GenerateDescription}{g} \\
   candidates \GETS \CALL{Ontology.Find}{\mathcal{D}} \\
   
   \IF \left|{candidates}\right| = 0 \THEN
    \BEGIN
      \OUTPUT{\mbox{Couldn't resolve the group!}} \\
      \EXIT \\
    \END
   \ELSEIF \left|{candidates}\right| = 1 \THEN
      id \GETS candidates[0]
   \ELSE
      \BEGIN
        \IF \CALL{Ontology.CheckEquivalent}{candidates} \THEN
          id \GETS candidates[0] \\
        \ELSE
          id \GETS \CALL{Discrimination}{candidates} \STMTNUM{1em}{st.discrimination}\\
      \END \\
   \CALL{Replace}{g, id, sentence}
\END
\ENDMAIN
\end{pseudocode}


The discrimination routine called at (\ref{st.discrimination}) is described separately. 
Two remarks must be made that doesn't appear in alg.~\ref{Resolution}:
\begin{enumerate}
    \item If a sentence starts with {\it Learn that...}, failures during 
    discrimination are interpreted as new concepts, and instead of marking the 
    nominal as not resolved, and new identifier is created.
    \item For questions like {\it Which color is the bottle?}, the discrimination 
    algorithm can not use the feature {\it color} to identify to bottle. The 
    resolution algorithm pass this kind of constraints as a parameter of the 
    \sc{Discrimination} routines.
\end{enumerate}

The method called at (\ref{st.history}) consists in looking through sentences that have been stored in the conversation history, then extracting their nominal group in order to retrieve the identifier of the most recently mentioned concept that holds the same characteristics as the nominal group that is being processed.

\end{document}
